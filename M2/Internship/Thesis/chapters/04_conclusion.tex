\label{chap:conclusions}

\begin{enumerate}[wide=0pt]
% \section{On the use of ontologies}

\item Ontologies are used in different applications with mainly the unified goal of having a shared vocabulary 
with agreed-upon semantics. Such applications have emerged from domains such as fraud detection, 
web mining, search engines, legacy system integration, e-learning, data-level data integration, semantic 
publishing, and so on \cite{cmariakeet,ontotext}. You can imagine how having a shared vocabulary could 
potentially help to create more easily integratable systems and therefore, an ecosystem. Ontologies are 
very useful for such purposes.

\item Although developing and using ontologies can be very helpful for many real-world applications, they can 
also be painful to deal with in other areas of interest. The first problem is that it is not probably 
practically possible to define everything very rigorously by using some subset of the first-order logic. 
For example, how could one try to solve an image classification problem with the help of an ontology, or 
play chess, or solve a maze? Such problems require different solutions than using pure ontologies.
The second issue with using ontologies is that they have their own limitations in terms of the 
expressivity. To make the inference process tractable and decidable, there are different types of 
\textit{profiles} that are suitable for different use cases. The trade-off is obvious - more expressivity, 
less tractablity. For example, if you worked with the RDF framework, it would be impossible for you to 
express ``a class being a subclass of some another class'' since there is no such object property 
defined, and to be able to do it, RDFS would be one of the options to choose. However, as mentioned 
earlier, every framework has the trade-off between expressivity and tractability.

% \section{On the use of OTTR}

\item The idealogy behind OTTR is simple and powerful - \textit{using templates to build knowledge bases with 
better readability, maintainability and usability}. What it gives to its users is a more efficient 
process 
of dealing with ontologies and knowledge bases. Templates used in OTTR language prevent many potential 
repetitions by making the whole process more modular. Such separation between the design and the content 
allows one to develop a template library once and use it as much as needed to add new instances to the 
knowledge base.

Serialization of OTTR templates into RDF/OWL format is another benefit of using them. Imagine you have 
built a template library that can be used to add instances to the KB and the library itself could be 
loaded to the same KB. You would not need to store the library file separately on your local machine or 
some separate database than the one which contains the instances. So, the library file can be fetched 
from the same database either to check the validity of the triples or add new instances more reliably and 
easily.

% \section{Experiences gained during the internship}

\item To be able to solve the knowledge acquisition task, we encountered many problems along the way. These 
problems included thinking individually and brainstorming as a team with my supervisor. Communication 
and shared vocabulary played an important role during such times since when having a discussion about 
ontologies, people can easily misunderstand each other due to ambiguously defined and/or used words. 
However, because of such situations the need for rigor became very obvious to me. I understood the 
non-triviality of building a good ontology, building trustfulness by centeralization\footnote{KG has 
been built to play the role of single unified information source.}, decreasing performance issues 
when scaling up the database size, etc. These real-world problems made me realize the importance of 
scientific methods, differences between theory(thinking without constraints) and practice(engineering 
- thinking with constraints), communication and organization.
\end{enumerate}

% \section{Side notes}
% 
% Artificial Intelligence
