LITIS is a laboratory of the University of Havre Normandy, INSA Rouen Normandy and University of Rouen 
Normandy. The laboratory is a member of ``Digital Normandy'' which is the Norman network and the 
doctoral school of MIIS. LITIS is also a partner of the Normastic CNRS Research Federation. 
There are several research fields in LITIS which are shown below:

\begin{itemize}
	\item Information access
	\item Ambient intelligence
	\item Biomedical information processing
\end{itemize}

Beside the above shown fields of research, there are also several applications that are being 
addressed to by the work of laboratory workers:

\begin{itemize}
	\item Automotive and smart territories
	\item Information acces in all sectors
	\item Health
\end{itemize}

There are several departments in the laboratory each of which has expertise and works on different 
fields and aims at professional quality results. These fields are as shown below:

\begin{itemize}
	\item \textbf{Machine Learning}\newline
		Different techniques and/or models such as Markovian models, graph-based classification, 
		kernel machines, etc. are used in ML applications. Understanding and being able to use relevant 
		strategies is not a trivial job and the laboratory has theoretical and algorithmic expertise 
		on these domains.
	\item \textbf{Intelligent vehicle}\newline
		Transportation can become even efficient with the help of intelligent vehicles and other tools 
		used in everyday transit. Big data management for such intelligent transportation systems 
		plays an important role for this field of research.
	\item \textbf{Multi-agent systems}\newline
		Use of AI and Semantic Web technologies in order to automate decision-making processes through 
		reasoning and explainability is the main goal. This research field also focuses on the 
		human-machine interfaces for the development of socio-technical systems.
	\item \textbf{Health and Information Technology}\newline
		Analyzing biologcial data to gather relevant information in the field of bioinformatics. 
		Therapeutic monitoring and prediction in the domain of medical imaging is also a part of this 
		research field.
	\item \textbf{Combinatorics and Algorithms}\newline
		Having many applications in information processing such as words, automatons, free monoids, 
		etc., the members of this field focus on the study of models of algebraic nature to analyze 
		combinatorial and algorithmic aspects of them.
\end{itemize}

Besides scientific aspects and achievements of the LITIS laboratory, there are safety guidelines 
for emergency and those guidelines are practiced by the workers from time to time. In LITIS, 
researchers are very welcoming and they usually try to help each other on local social media channels 
when there is some technical problems to be solved. In my opinion, such attitude is one of the biggest 
achievements of the laboratory or even any company.

% The laboratory keeps close relations with Malaysia, Argentina, Canada, Brazil, China, Spain, 
% South Korea, The United States, Romania, Morocco, Italy and so on.
% LITIS also collaborates with many leading international groups such as Orange Labs, Airbus Defense and 
% Space, ITESOFT, bioMérieux, Peugeot PSA, Siemens, Valeo as well as some other small companies.

% \section{Interview}

% This interview has been taken from ??? who is currently an engineer working on ??? in the LITIS laboratory.

% \begin{table}[ht]
% 	\centering
% 	\begin{tabular}{p{0.1\textwidth}p{0.9\textwidth}}
% 		\hline
% 		\textbf{Q:} & What is intelligence? (is it reproducable by computation?) \\
% 		\textbf{A:} & ? \\
% 		\hline
% 		\textbf{Q:} & What is consciousness? (is it reproducable by computation?) \\
% 		\textbf{A:} & ? \\
% 		\hline
% 		\textbf{Q:} & We usually talk about the need for black box models to be explainable and so forth. 
% 		Up until now, full functionality of our brains have not been discoevered completely and there 
% 		are many puzzles waiting to be solved in the future. How do you think we ara capable of explaining 
% 		things without even acknowledging what is going on underneath? If our explainations are completely 
% 		the by-product of our brain activities, do you think that explainability could potentially emerge 
% 		from black-box or non-explainable models at some point? \\
% 		\textbf{A:} & ? \\
% 		\hline
% 		\textbf{Q:} & The problem-solving methodologies are developed either to deal with continuous or 
% 		discrete spaces. For example, problems related to Integer Programming, Boolean Satisfiability, 
% 		Domain-specific Language Search are inherently discrete whereas problems related to Linear 
% 		Regression, Artificial Neural Networks, Kernel Machines are continious by nature. What do you 
% 		think of the dichotonomy between ``discrete'' and ``continuous''? \\
% 		\textbf{A:} & ? \\
% 		\hline
% 		\textbf{Q:} & How do you think more generalization power could be achieved by the AI models 
% 		and do you think of (lossless) compression as generalization?
% 		\textbf{A:} & ? \\
% 		\hline
% 		\textbf{Q:} & If you had to classify your thoughts/works in AI, what category would you choose - 
% 		evolutionist, emergentist/reductionist, symbolist, connectionist, Bayesianist?
% 		\textbf{A:} & ? \\
% 		\hline
% 		\textbf{Q:} & What pieces do connectionism (i.e., ML/DL) lack in order to fully satisfy our 
% 		needs or achieve better problem solving abilities? \\
% 		\textbf{A:} & ? \\
% 		\hline
% 		\textbf{Q:} & What pieces do symbolism (i.e., expert systems) lack in order to fully satisfy 
% 		our needs or achieve better problem solving abilities? \\
% 		\textbf{A:} & ? \\
% 		\hline
% 		\textbf{Q:} & What type of architectures do you think will be used the most in the future? \\
% 		\textbf{A:} & ? \\
% 		\hline
% 		\textbf{Q:} & What would be your advice to the people working on your field? \\
% 		\textbf{A:} & ? \\
% 		\hline
% 	\end{tabular}
% 	\caption{Interview Q\&A}
% 	\label{tab:interview}
% \end{table}
