\documentclass[a4paper]{article}
\usepackage[utf8]{inputenc}

\usepackage{amsthm}
\usepackage{amssymb}
\usepackage{amsmath}
\usepackage{mathtools}

\usepackage[ruled,vlined]{algorithm2e} 
% \usepackage{algorithm}
% \usepackage{algorithmic}
\usepackage{array}
\usepackage{listings}
\usepackage{multirow}
        
\usepackage[dvipsnames]{xcolor}
\usepackage[toc,page]{appendix}
\usepackage{tikz}
\usepackage{float}
\usepackage{graphicx}
\usepackage{caption} 
\usepackage{subcaption}
\usepackage[colorlinks = true, 
            linkcolor = blue,
            urlcolor  = blue, 
            citecolor = blue,
            anchorcolor = blue]{hyperref}
        
\graphicspath{ {images/} }
\usepackage[a4paper,width=165mm,top=22mm,bottom=22mm]{geometry}
\setlength{\headheight}{15pt}
% \usepackage[most]{tcolorbox}
        
\title{Developing and Comparing CNN and RNN for Time Series Classification}
\author{Ali Khudiyev}
\date{November 2021}

\begin{document}
\maketitle

\begin{abstract}
	In this paper, CNN and RNN models are developed, tested and evaluated in terms of better accuracy, precision, recall and f1-score.
\end{abstract}

\section{Introduction}


\section{Development}
\subsection{Developing CNN}

\subsection{Developing RNN}

\section{Experiments}
\subsection{CNN}
\subsection{RNN}

\section{Results \& Conclusion}

\end{document}
